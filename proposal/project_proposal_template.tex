\documentclass[12pt]{article}
\usepackage[margin=1in]{geometry}
\usepackage{hyperref}

\setlength{\parindent}{15px}

\title{Exoplanet Analysis - NASA Data}

\author{Tom Wilson, Ben Irving}

\date{10 - 22 - 2021}

\begin{document}

\maketitle

\abstract
The purpose of the following analysis is to discern patterns in the collected Kepler data currently available on confirmed exoplanets, to formulate a better understanding of planetary conditions and how they might correlate to one another. The object would be to gain a new perspective on how these conditions might lead to environments suitable for life, and other important physical processes. 
\section{Introduction}
\paragraph*{}

 \indent The first exoplanets were confirmed in 1992, following the discovery made by Aleksander Wolszczan and Dale Frail. Since then, over 3600 planets have been formalized by various observatories and telescopes throughout Space. The study of these planetary bodies becomes increasingly relevant, as humanity stands on the edge entering the cosmic arena in full. As such, the study of conditions on other planets will help inform humanity of ways in which more local environments may evolve, on Earth and other bodies in the Solar System. 
\\
\indent The field is new, with the first exoplanet being identified less than 30 years ago. As such, there is very little known about how planets emerge in other areas of the universe, and how their physical properties develop with respect to their stellar bodies. \\
\indent NASA Space Telescopes like the Kepler and recently retired Spitzer track large amounts of relevant information about the gravitational properties, light \& wave effects, and other common metrics used to study and classify the hidden stellar objects. Much of this information is available to us free of cost thanks to the wonders of the internet. \\
\indent With said data, one can identify conditions which could potentially lend themselves to life and other persistent phenomena. Of course, no conclusions can be made about the data, as none of the planets are close enough for thorough observation, yet we can infer some ideas based off of what we know about the planets in our own solar system, particularly Earth. Liquid water, for instance, appears to be a central component to the emergence of life. Yet, perhaps there exists some form of existence which does not depend on water, but we would not be able to conceive of its survival means because we have not yet encountered such a creature. \\
\indent That being said, in terms of relative habitability, we plan to move forwards with the data available and the knowledge that currently exists surrounding the nature of life to infer conclusions about certain planets, sorting them into the four categories listed below. Beyond capturing personal interest, identifying patterns in such data is a problem that many of the foremost institutions around the world are embarking upon, to better understand the nature of the cosmos. 
\section{Proposed Project}
4 classes: 
Not habitable | Semi-Habitable | Habitable | Superhabitable*

(possible characteristics to classify by: gravity, planetary density, stellar properties 
(any possible atmospheric metrics)
(age, size, orbit))

Key Factors about the exoplanet's data can be correlated with one another using regressions via Gradient descent.
\\
\indent We plan to regress on atmospheric density, stellar density, stellar age (and a assortment of other features) to create a model which can accurately predict planetary density. Planetary statistics, in terms of long range observations, are the hardest to come by, as most of them are observed as shadows in front of stars. Thus, finding any relevant patterns pertaining to planetary information would be fascinating. \\
\indent Furthermore, we would regress on available planetary/stellar data, to find average temperature, an extremely important characteristic when it comes to identifying environments suitable to life. 
\\
\indent Initial approach may be to establish factors of high co variance in order to rank factors in terms of their correlation with feature values which we recognize to be conducive (or not) to habitability on our mother Earth and our surrounding planets.

*Because it is perhaps anthropocentric and naive to consider habitability a function of similarity to Earth, other factors may be included in a special class of planets (though an unsupervised approach may fit this task better)

As a side pursuit, we would like to train a neural network able to build ideal planetary bodies ("super habitable" environments) by listing characteristics based on stellar properties. Given a system, we would build an "ideal" planet, in terms of distance from the star and other such aspects.

\section{Sources}

\begin{itemize}
\item[{\bf 1.)}] \textbf{ NASA Exoplanet archive} \\
\emph{https://exoplanetarchive.ipac.caltech.edu/docs/data.html}
\begin{itemize}
\item Planetary System Information
\item Transmission \& Emission Spectroscopy
\item 
\end{itemize}
\end{itemize}


\end{document}
