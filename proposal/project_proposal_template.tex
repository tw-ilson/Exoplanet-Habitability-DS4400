\documentclass[12pt]{article}
\usepackage[margin=1in]{geometry}
\usepackage{hyperref}



\title{Exoplanet Analysis - NASA Data}

\author{Tom Wilson, Ben Irving}

\date{10 - 22 - 2021}

\begin{document}

\maketitle

\abstract
The purpose of the following analysis is to discern patterns in the collected Kepler data currently available on confirmed exoplanets, to formulate a better understanding of planetary conditions and how they might correlate to one another. The object would be to gain a new perspective on how these conditions might lead to environments suitable for life, and other important physical processes. 
\section{Introduction}
The first exoplanets were confirmed in 1992, following the discovery made by Aleksander Wolszczan and Dale Frail. Since then, over 3600 planets have been formalized by various observatories and telescopes throughout Space. The study of these planetary bodies becomes increasingly relevant, as humanity stands on the edge entering the cosmic arena in full. As such, the study of conditions on other planets will help inform humanity of ways in which more local environments may evolve, on Earth and other bodies in the Solar System. 
\\
The field is new, with the first exoplanet being identified less than 30 years ago. As such, there I very little known about how planets emerge in other areas of the universe, and how their physical properties develop with respect to their stellar bodies. \\
NASA Space Telescopes like the Kepler and recently retired Spitzer track large amounts of relevant information about the gravitational properties, light \& wave effects, and other common metrics used to study and classify the hidden stellar objects. Much of this information is available to us free of cost thanks to the wonders of the internet.
\section{Proposed Project}
4 classes: 
Not habitable | Semi-Habitable | Habitable | Superhabitable*

(possible characteristics to classify by: gravity, planetary density, stellar properties 
(any possible atmospheric metrics)
(age, size, orbit))

Key Factors about the exoplanet's data can be correlated with one another using regressions via Gradient descent

Initial approach may be to establish factors of high covariance in order to rank factors in terms of their correlation with feature values which we recognize to be conducive (or not) to habitability on our mother Earth and our surrounding planets.

*Becuase it is perhaps anthropocentric and naiive to consider habitability a function of similarity to Earth, other factors may be included in a special class of planets (though an unsupervised approach may fit this task better)

\section{Sources}

\begin{itemize}
\item[{\bf 1.)}] \textbf{ NASA Exoplanet archive} \\
\emph{https://exoplanetarchive.ipac.caltech.edu/docs/data.html}
\begin{itemize}
\item Planetary System Information
\item Transmission \& Emission Spectroscopy
\item 
\end{itemize}
\end{itemize}


\end{document}
